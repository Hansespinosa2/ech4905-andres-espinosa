\documentclass[conference]{IEEEtran}
\usepackage{xcolor}
\usepackage{url}
\usepackage{graphicx} 
\usepackage{amssymb}
\usepackage{amsmath}

\usepackage{listings}
\usepackage{xcolor}

\definecolor{codegreen}{rgb}{0,0.6,0}
\definecolor{codegray}{rgb}{0.5,0.5,0.5}
\definecolor{codepurple}{rgb}{0.58,0,0.82}
\definecolor{backcolour}{rgb}{0.97,0.97,0.97}

\lstdefinestyle{mypython}{
    backgroundcolor=\color{backcolour},   
    commentstyle=\color{codegreen},
    keywordstyle=\color{blue}\bfseries,
    numberstyle=\tiny\color{codegray},
    stringstyle=\color{codepurple},
    basicstyle=\ttfamily\footnotesize,
    breakatwhitespace=false,         
    breaklines=true,                 
    captionpos=b,
    keepspaces=true,                 
    numbers=left,                    
    numbersep=5pt,                  
    showspaces=false,                
    showstringspaces=false,
    showtabs=false,                  
    tabsize=4,
    language=Python
}

\hyphenation{op-tical net-works semi-conduc-tor}
\definecolor{uf_blue}{RGB}{17,27, 150}
\definecolor{uf_orange}{RGB}{150,100,17}

\begin{document}

\title{GatorPy: A Custom Implemented Linear Programming Solver}

\author{
    \begin{minipage}{0.50\textwidth}
        \centering
        \textcolor{uf_blue}{Andres Espinosa} \\
        \textcolor{uf_orange}{Industrial and Systems Engineering} \\
        \textcolor{uf_orange}{University of Florida} \\
        \textcolor{uf_orange}{andresespinosa@ufl.edu} \\ 
    \end{minipage}
}


\maketitle


\begin{abstract}

\end{abstract}

\IEEEpeerreviewmaketitle

\section{Introduction}
GatorPy is a custom Linear Programming solver implemented entirely in Python.
The purpose of this project is twofold: 
First, to serve as an educational tool as an example of a simple and custom implementation of an LP solver. 
Second, to establish a foundation for other students to build upon and contribute to a collaborative open-source University of Florida custom optimization solver.

\subsection{Linear Programming}


\subsection{Simplex Algorithm}

\subsection{Available Solvers}
There are numerous available optimization solvers, both commercial and open-source.
One particular source of inspiration for this project is CVXPY, an open-source convex optimization solver \cite{solvers:diamond2016cvxpy}.


\section{Implementation}

\subsection{GatorPy Syntax}
The overarching goal with the optimization modeling syntax is to maintain a healthy balance between a pythonic syntax and standard optimization linear algebra notation.
GatorPy relies heavily on the NumPy numerical processing package in Python.
The general structure of a GatorPy problem involves the following steps:
\begin{enumerate}
    \item Create \texttt{Parameter} objects for each parameter in the problem.
    Each \texttt{Parameter} object takes in a \texttt{np.array} as the value.
    \item Create \texttt{Variable} objects for each variable in the problem.
    Each \texttt{Variable} takes in an integer as the shape of the vector.
    \textit{Note: Each variable must be a vector, this is left as a potential next step in section \ref{future_work}}.
    \item Create a \texttt{Problem} object representing the overall problem.
    The \texttt{Problem} object expects a Python \texttt{dict} object with the following key-value pairs:
    \begin{itemize}
        \item Either \texttt{"minimize"} or \texttt{"maximize"} as a key with a GatorPy \texttt{Expression} as the value.
        \item Either \texttt{"subject to"} or \texttt{"constraints"} as a key with a list of GatorPy \texttt{Constraint} objects as the value.
    \end{itemize}
\end{enumerate}
The simple syntax of GatorPy can be best communicated with an example.
Consider the following optimization problem with two variables and three constraints.
\begin{align*}
  \text{maximize} & \quad \textbf{c}^\top \textbf{y} \\
  \text{subject to} & \quad \textbf{A} \textbf{y} \preceq \textbf{b} \\
  & \quad \textbf{y} \preceq \textbf{1} \\
  & \quad \textbf{y} \succeq \textbf{0}
\end{align*}
where
\begin{align*}
    \textbf{c} = \begin{bmatrix} 1.2 \\ 0.5 \end{bmatrix}, \quad
    \textbf{A} = \begin{bmatrix} 1 & 1 \\ 1.2 & 0.5 \end{bmatrix}, \quad
    \textbf{b} = \begin{bmatrix} 1 \\ 1 \end{bmatrix}, \quad
    \textbf{y} = \begin{bmatrix} y_1 \\ y_2  \end{bmatrix}
\end{align*}

This above optimization problem can be expressed in GatorPy as:
\begin{lstlisting}[style=mypython, caption={Solving a Linear Program Symbolically}]
# Parameters
A = Parameter(np.array([[1,1],[1.2,0.5]]))
b = Parameter(np.array([1,1]))
c = Parameter(np.array([1.2,1]))

# Variables
y = Variable(2)

# Problem
problem = Problem({
    'maximize': c.T @ y,
    'subject to': [
        A @ y <= b,
        y <= 1,
        y >= 0
    ]
})

solution = problem.solve()
print(solution)
>>> (1.14, [0.71, 0.29], True)
\end{lstlisting}

\subsection{Simplex Implementation}
\subsection{Python Objects}

\subsection{LP Reductions}

\section{Results}
\subsection{Testing Framework}

\subsection{Testing Results}


\section{Discussion}
\subsection{Future Work}
\label{future_work}

\subsection{Conclusion}




\bibliographystyle{IEEEtran}
\bibliography{references}  % Assuming your .bib file is named references.bib


\end{document}



