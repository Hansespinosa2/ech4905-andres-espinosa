\documentclass[conference]{IEEEtran}
\usepackage{xcolor}
\usepackage{url}
\usepackage{graphicx} 
\usepackage{amssymb}
\usepackage{amsmath}
\hyphenation{op-tical net-works semi-conduc-tor}
\definecolor{uf_blue}{RGB}{17,27, 150}
\definecolor{uf_orange}{RGB}{150,100,17}

\begin{document}

\title{Learning to Dispatch: A Reinforcement Learning Framework for Train Networks} %Should think about this at the meeting more.

\author{
    \begin{minipage}{0.50\textwidth}
        \centering
        \textcolor{uf_blue}{Andres Espinosa} \\
        \textcolor{uf_orange}{Industrial and Systems Engineering} \\
        \textcolor{uf_orange}{University of Florida} \\
        \textcolor{uf_orange}{andresespinosa@ufl.edu} \\ 
    \end{minipage}
}


\maketitle


\begin{abstract}

\end{abstract}

\IEEEpeerreviewmaketitle

\section{Introduction}

\section{Problem Background}

\subsection{Train Dispatch Problem}
% This should cover the MIP formulation and the general outline of the problem
% This should also cover the DISPLIB data format and all that.
\subsection{Deep Reinforcement Learning}
% Cover Markov Decision Processes
% As well as DQN
\subsection{Graph Neural Networks}
% Cover graph neural networks

\section{Related Work}
% This can be a full literature review following the structure of:
% 1) Train Dispatch Problem Formulation
% 2) Deep Reinforcement Learning with similar problems
% 3) Graph Neural Networks
I am \cite{gnndrl:Devailly_2022}
\section{Implementation}

\subsection{State Space}
\subsection{Action Space}

\section{Results}


\section{Conclusion and Future Work}

\section{Contributions}

\section{Acknowledgements}

\section{Appendix}




\bibliographystyle{IEEEtran}
\bibliography{references}  % Assuming your .bib file is named references.bib


\end{document}



