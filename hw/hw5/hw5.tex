\RequirePackage[orthodox]{nag}
\documentclass[11pt]{article}

%% Define the include path
\makeatletter
\providecommand*{\input@path}{}
\g@addto@macro\input@path{{include/}{../include/}}
\makeatother

\usepackage{../../include/akazachk}


\title{ECH4905 ChemE Optimization HW 4}
\author{Andres Espinosa}

\begin{document}
\maketitle

\section{Problem 1}
Consider the following integer programming problem:


\begin{align*}
  \text{maximize} & \quad 1.2y_1 + y_2 \\
  \text{subject to} & \quad y_1 + y_2 \leq 1 \\
  & \quad 1.2y_1 + 0.5y_2 \leq 1 \\
  & \quad y_1, y_2 \in \{ 0,1 \}
\end{align*}

\subsection{Part a}
\label{prob1parta}
Solve the first relaxed LP subproblem by hand using the simplex method and derive Gomory cuts based on the LP relaxation.

\textbf{Solution:} To tackle this problem we first relax the problem and then turn the problem above into the standard form so we can create a simplex tableau from it.

\begin{align*}
    \text{maximize} & \quad 1.2y_1 + y_2 \\
    \text{subject to} & \quad y_1 + y_2 + s_1= 1 \\
    & \quad 1.2y_1 + 0.5y_2 + s_2 = 1 \\
    & \quad y_1 + s_3 = 1 \\
    & \quad y_2 + s_4 = 1 \\
    & \quad y_1, y_2, s_1, s_2, s_3, s_4 \geq 0
\end{align*}

In matrix notation,
\begin{align*}
  \text{minimize} & \quad \textbf{c}^\top \textbf{x} \\
  \text{subject to} & \quad \textbf{A} \textbf{x} = \textbf{b} \\
  & \quad \textbf{x} \succeq 0
\end{align*}
where
\begin{align*}
    \textbf{c} = 
  \begin{bmatrix}
     -1.2 \\ -1 \\ 0 \\ 0 \\ 0 \\ 0
  \end{bmatrix}, \quad
  \textbf{A} = 
  \begin{bmatrix}
    1 & 1 & 1 & 0 & 0 & 0 \\
    1.2 & 0.5 & 0 & 1 & 0 & 0 \\
    1 & 0 & 0 & 0 & 1 & 0 \\
    0 & 1 & 0 & 0 & 0 & 1
  \end{bmatrix}, \quad
  \textbf{b} = 
  \begin{bmatrix}
    1 \\ 1 \\ 1 \\ 1
  \end{bmatrix}, \quad
  \textbf{x} = 
  \begin{bmatrix}
    y_1 \\ y_2 \\ s_1 \\ s_2 \\ s_3 \\ s_4
  \end{bmatrix}
\end{align*}
(with a flipped objective component)

The initial simplex tableau for the problem is as follows:

\[
\begin{array}{c|cccccc|c|c}
\text{Basic Var} & y_1 & y_2 & s_1 & s_2 & s_3 & s_4 & \text{RHS} & \alpha \\
\hline
s_1 & 1 & 1 & 1 & 0 & 0 & 0 & 1 &  \\
s_2 & 1.2 & 0.5 & 0 & 1 & 0 & 0 & 1 &  \\
s_3 & 1 & 0 & 0 & 0 & 1 & 0 & 1 &  \\
s_4 & 0 & 1 & 0 & 0 & 0 & 1 & 1 &  \\
\hline
\text{obj} & -1.2 & -1 & 0 & 0 & 0 & 0 & - & - \\
\end{array}
\]
we can define the slack variables equal to the right hand side, and this is in turn a basic feasible solution, so we can jump into phase 2.

We select the $y_1$ as the entering variable and calculate the alpha value for each basic variable
\[
\begin{array}{c|cccccc|c|c}
\text{Basic Var} & y_1 & y_2 & s_1 & s_2 & s_3 & s_4 & \text{RHS} & \alpha \\
\hline
s_1 & 1 & 1 & 1 & 0 & 0 & 0 & 1 & \frac{1}{1} \\
s_2 & 1.2 & 0.5 & 0 & 1 & 0 & 0 & 1 & \frac{1}{1.2} \\
s_3 & 1 & 0 & 0 & 0 & 1 & 0 & 1 & \frac{1}{1} \\
s_4 & 0 & 1 & 0 & 0 & 0 & 1 & 1 & \frac{1}{0} \\
\hline
\text{obj} & -1.2 & -1 & 0 & 0 & 0 & 0 & - & - \\
\end{array}
\]
We pivot this on the 1st column ($y_1$) and the 2nd row ($s_2$)
\[
\begin{array}{c|cccccc|c|c}
\text{Basic Var} & y_1 & y_2 & s_1 & s_2 & s_3 & s_4 & \text{RHS} & \alpha \\
\hline
s_1 & 0 & \frac{7}{12} & 1 & -\frac{5}{6} & 0 & 0 & \frac{1}{6} &  \\
y_1 & 1 & \frac{5}{12} & 0 & \frac{5}{6} & 0 & 0 & \frac{5}{6} &  \\
s_3 & 0 & -\frac{5}{12} & 0 &  -\frac{5}{6} & 1 & 0 & \frac{1}{6} &  \\
s_4 & 0 & 1 & 0 &  0 & 0 & 1 & 1 &  \\
\hline
\text{obj} & 0 & -0.5 & - & - & - & - & - & - \\
\end{array}
\]
With blands rule, we pick $y_2$ and calculate the alpha value for each basic variable.
\[
\begin{array}{c|cccccc|c|c}
\text{Basic Var} & y_1 & y_2 & s_1 & s_2 & s_3 & s_4 & \text{RHS} & \alpha \\
\hline
s_1 & 0 & \frac{7}{12} & 1 & -\frac{5}{6} & 0 & 0 & \frac{1}{6} & \frac{2}{7} \\
y_1 & 1 & \frac{5}{12} & 0 & \frac{5}{6} & 0 & 0 & \frac{5}{6} & \frac{2}{1} \\
s_3 & 0 & -\frac{5}{12} & 0 &  -\frac{5}{6} & 1 & 0 & \frac{1}{6} & -\frac{1}{5} \\
s_4 & 0 & 1 & 0 &  0 & 0 & 1 & 1 & \frac{1}{1} \\
\hline
\text{obj} & 0 & -0.5 & - & - & - & - & - & - \\
\end{array}
\]
We pivot on the 2nd column ($y_2$) and the 1st row ($s_1$).
\[
\begin{array}{c|cccccc|c|c}
\text{Basic Var} & y_1 & y_2 & s_1 & s_2 & s_3 & s_4 & \text{RHS} & \alpha \\
\hline
y_2 & 0 & 1 & \frac{12}{7} & -\frac{10}{7} & 0 & 0 & \frac{2}{7} &  \\
y_1 & 1 & 0 & -\frac{5}{7} & \frac{60}{42} & 0 & 0 & \frac{30}{42} &  \\
s_3 & 0 & 0 & \frac{5}{7} &  -\frac{60}{42} & 1 & 0 & \frac{12}{42} &  \\
s_4 & 0 & 0 & -\frac{12}{7} &  \frac{60}{42} & 0 & 1 & \frac{30}{42} &  \\
\hline
\text{obj} & 0 & 0 & 0.857 & 0.286  & 0 & 0 & 0 & -1.143 \\
\end{array}
\]
This is the optimal solution to the LP relaxed problem.
Now we will derive Gomory cuts from this LP relaxed problem.
Since each constraint has a non-integer solution, we can generate a Gomory cut on each constraint.

\begin{align*}
  y_2 + \text{floor}(\frac{12}{7}) s_1 + \text{floor}(\frac{-10}{7}) s_2 \leq \text{floor}(\frac{2}{7}) \\
  y_1 + \text{floor}(\frac{-5}{7}) s_1 + \text{floor}(\frac{10}{7}) s_2 \leq \text{floor}(\frac{5}{7}) \\
  s_3 + \text{floor}(\frac{5}{7}) s_1 + \text{floor}(\frac{-10}{7})s_2 \leq \text{floor}(\frac{2}{7}) \\
  s_4 + \text{floor}(\frac{-12}{7}) s_1 + \text{floor}(\frac{10}{7}) s_2 \leq \text{floor}(\frac{5}{7})
\end{align*}
These turn into the cuts
\begin{align*}
  y_2 + s_1 -2 s_2 \leq 0 \\
  y_1 - s_1 + s_2 \leq 0 \\
  s_3 -2 s_2 \leq 0 \\
  s_4 -2 s_1 + s_2 \leq 0
\end{align*}

\subsection{Part b}
Solve the above problem with the branch and bound method by enumerating nodes in the tree and solving the LP subproblems using GAMS.

\textbf{Solution: }
The initial LP relaxed problem is solved in \ref{prob1parta}, so we can start with the parent node.
An important note for this question, I will be solving the LPs in my custom \texttt{gatorpy} LP solver so that I can use them as verification tests.
The code used will be available in section 
\begin{center}
  \begin{adjustbox}{width=0.5\textwidth}
  \begin{tikzpicture}[
    level 1/.style={sibling distance=80mm},
    level 2/.style={sibling distance=35mm},
    level distance=3.5cm,
    edge from parent/.style={draw, -latex, very thick},
    every node/.style={draw, rounded corners, text width=10cm, align=center, very thick, font=\small}
]
    
% Root node
\node {
  \begin{align*}
      & [\frac{5}{7},  \frac{2}{7},0,0,\frac{2}{7},\frac{5}{7}], z = \frac{8}{7} \\
      & \text{maximize} \quad 1.2y_1 + y_2 \\
      & \text{subject to} \quad y_1 + y_2 + s_1 = 1, \\
      & \quad 1.2y_1 + 0.5y_2 + s_2 = 1, \\
      & \quad y_1 + s_3 = 1, \\
      & \quad y_2 + s_4 = 1, \\
      & \quad y_1, y_2, s_1, s_2, s_3, s_4 \geq 0. \\
      & UB= \frac{8}{7}, LB = -inf
  \end{align*}
};
\end{tikzpicture}
\end{adjustbox}
\end{center}
Since all variables are fractional, we can pick the first one $y_1$ to branch on
\begin{center}
  \begin{adjustbox}{width=0.8\textwidth}
  \begin{tikzpicture}[
    level 1/.style={sibling distance=100mm},
    level 2/.style={sibling distance=80mm},
    level distance=10cm,
    edge from parent/.style={draw, -latex, very thick},
    every node/.style={draw, rounded corners, text width=10cm, align=center, very thick, font=\small}
]
    
% Root node
\node {
  \begin{align*}
      & [\frac{5}{7},  \frac{2}{7},0,0,\frac{2}{7},\frac{5}{7}], z = \frac{8}{7} \\
      & \text{maximize} \quad 1.2y_1 + y_2 \\
      & \text{subject to} \quad y_1 + y_2 + s_1 = 1, \\
      & \quad 1.2y_1 + 0.5y_2 + s_2 = 1, \\
      & \quad y_1 + s_3 = 1, \\
      & \quad y_2 + s_4 = 1, \\
      & \quad y_1, y_2, s_1, s_2, s_3, s_4 \geq 0. \\
      & UB= \frac{8}{7}, LB = -inf
  \end{align*}
}
    % Left child of Root
    child {node {
      \begin{align*}
          & [0, \frac{2}{3}, 0, \frac{1}{3}, \frac{1}{3}, 1], z = \frac{2}{3} \\
          & \text{maximize} \quad 1.2y_1 + y_2 \\
          & \text{subject to} \quad y_1 \leq 0, \\
          & \quad y_1 + y_2 + s_1 = 1, \\
          & \quad 1.2y_1 + 0.5y_2 + s_2 = 1, \\
          & \quad y_1 + s_3 = 1, \\
          & \quad y_2 + s_4 = 1, \\
          & \quad y_1, y_2, s_1, s_2, s_3, s_4 \geq 0. \\
          & UB= \frac{2}{3}, LB = -inf
      \end{align*}
    }
    edge from parent node[draw=none, pos=0.5, above left] {$y_1 = 0$}}
    % Right child of Root
    child {node {
      \begin{align*}
          & [1, 0, 0, 0, 0, 1], z = 1.2 \\
          & \text{maximize} \quad 1.2y_1 + y_2 \\
          & \text{subject to} \quad y_1 \geq 1, \\
          & \quad y_1 + y_2 + s_1 = 1, \\
          & \quad 1.2y_1 + 0.5y_2 + s_2 = 1, \\
          & \quad y_1 + s_3 = 1, \\
          & \quad y_2 + s_4 = 1, \\
          & \quad y_1, y_2, s_1, s_2, s_3, s_4 \geq 0. \\
          & UB= 1.2, LB = -inf
      \end{align*}
    }
    edge from parent node[draw=none, pos=0.5, above right] {$y_1 = 1$}};
\end{tikzpicture}
\end{adjustbox}
\end{center}



% \begin{center}
%   \begin{adjustbox}{width=0.925\textwidth}
%   \begin{tikzpicture}[
%     level 1/.style={sibling distance=80mm},
%     level 2/.style={sibling distance=35mm},
%     level distance=3.5cm,
%     edge from parent/.style={draw, -latex, very thick},
%     every node/.style={draw, rounded corners, text width=2.5cm, align=center, very thick, font=\small}
% ]
    
% % Root node
% \node {\textbf{$\mathbf{{x \leq 7}}$}\\ Samples: 8 \\ MSE: 8.69 \\ $\hat y = 4.75$}
%     % Left child of Root
%     child {node {\textbf{$\mathbf{{x \leq 4}}$}\\ Samples: 4 \\ MSE: 1.69 \\ $\hat y = 2.25$}
%         % Left child of Node 1
%         child {node {\textbf{\underline{Node 3:}} \\ Samples: 2 \\ MSE: 0.00 \\ $\hat y = 1.00$}
%         edge from parent node[draw=none, pos=-0.4, above left, sloped] {{\color{blue}True}}}
%         % Right child of Node 1
%         child {node {\textbf{$\mathbf{{x \leq 5.5}}$} \\ Samples: 2 \\ MSE: 0.25 \\ $\hat y = 3.50$}
%             % Left child of Node 4
%             child {node {\textbf{\underline{Node 7:}} \\ Samples: 1 \\ MSE: 0.00 \\ $\hat y = 3.00$}
%             edge from parent node[draw=none, pos=-0.4, above left, sloped] {{\color{blue}True}}}
%             % Right child of Node 4
%             child {node {\textbf{\underline{Node 8:}} \\ Samples: 1 \\ MSE: 0.00 \\ $\hat y = 4.00$}
%             edge from parent node[draw=none, pos=-0.4, above right, sloped] {{\color{red}False}}
%             }
%         edge from parent node[draw=none, pos=-0.4, above right, sloped] {{\color{red}False}}}
%         edge from parent node[draw=none, pos=-0.2, above left, sloped] {{\color{blue}True}}
%     }
%     % Right child of Root
%     child {node {\textbf{$\mathbf{{x \leq 10.5}}$}\\ Samples: 4 \\ MSE: 3.19 \\ $\hat y = 7.25$}
%         % Left child of Node 2
%         child {node {\textbf{\underline{Node 5:}} \\ Samples: 2 \\ MSE: 0.0 \\ $\hat y = 9.00$}
%         edge from parent node[draw=none, pos=-0.4, above left, sloped] {{\color{blue}True}}}
%         % Right child of Node 2
%         child {node {\textbf{$\mathbf{{x \leq 12}}$} \\ Samples: 2 \\ MSE: 0.25 \\ $\hat y = 5.5$}
%             % Left child of Node 6
%             child {node {\textbf{\underline{Node 9:}} \\ Samples: 1 \\ MSE: 0.00 \\ $\hat y = 6.00$}
%             edge from parent node[draw=none, pos=-0.4, above left, sloped] {{\color{blue}True}}}
%             % Right child of Node 6
%             child {node {\textbf{\underline{Node 10:}} \\ Samples: 1 \\ MSE: 0.00 \\ $\hat y = 5.00$}
%             edge from parent node[draw=none, pos=-0.4, above right, sloped] {{\color{red}False}}
%             }
%         edge from parent node[draw=none, pos=-0.4, above right, sloped] {{\color{red}False}}}
%         edge from parent node[draw=none, pos=-0.2, above right, sloped] {{\color{red}False}}
%     };
% \end{tikzpicture}
% \end{adjustbox}
% \end{center}



\section{}

\end{document}