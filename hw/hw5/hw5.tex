\RequirePackage[orthodox]{nag}
\documentclass[11pt]{article}

%% Define the include path
\makeatletter
\providecommand*{\input@path}{}
\g@addto@macro\input@path{{include/}{../include/}}
\makeatother

\usepackage{../../include/akazachk}


\title{ECH4905 ChemE Optimization HW 4}
\author{Andres Espinosa}

\begin{document}
\maketitle

\section{Problem 1}
Consider the following integer programming problem:


\begin{align*}
  \text{maximize} & \quad 1.2y_1 + y_2 \\
  \text{subject to} & \quad y_1 + y_2 \leq 1 \\
  & \quad 1.2y_1 + 0.5y_2 \leq 1 \\
  & \quad y_1, y_2 \in \{ 0,1 \}
\end{align*}

\subsection{Part a}
Solve the first relaxed LP subproblem by hand using the simplex method and derive Gomory cuts based on the LP relaxation.

\textbf{Solution:} To tackle this problem we first relax the problem and then turn the problem above into the standard form so we can create a simplex tableau from it.

\begin{align*}
    \text{maximize} & \quad 1.2y_1 + y_2 \\
    \text{subject to} & \quad y_1 + y_2 + s_1= 1 \\
    & \quad 1.2y_1 + 0.5y_2 + s_2 = 1 \\
    & \quad y_1 + s_3 = 1 \\
    & \quad y_2 + s_4 = 1 \\
    & \quad y_1, y_2, s_1, s_2, s_3, s_4 \geq 0
\end{align*}

In matrix notation,
\begin{align*}
  \text{minimize} & \quad \textbf{c}^\top \textbf{x} \\
  \text{subject to} & \quad \textbf{A} \textbf{x} = \textbf{b} \\
  & \quad \textbf{x} \succeq 0
\end{align*}
where
\begin{align*}
    \textbf{c} = 
  \begin{bmatrix}
     -1.2 \\ -1 \\ 0 \\ 0 \\ 0 \\ 0
  \end{bmatrix}, \quad
  \textbf{A} = 
  \begin{bmatrix}
    1 & 1 & 1 & 0 & 0 & 0 \\
    1.2 & 0.5 & 0 & 1 & 0 & 0 \\
    1 & 0 & 0 & 0 & 1 & 0 \\
    0 & 1 & 0 & 0 & 0 & 1
  \end{bmatrix}, \quad
  \textbf{b} = 
  \begin{bmatrix}
    1 \\ 1 \\ 1 \\ 1
  \end{bmatrix}, \quad
  \textbf{x} = 
  \begin{bmatrix}
    y_1 \\ y_2 \\ s_1 \\ s_2 \\ s_3 \\ s_4
  \end{bmatrix}
\end{align*}
(with a flipped objective component)

The initial simplex tableau for the problem is as follows:

\[
\begin{array}{c|cccccc|c|c}
\text{Basic Var} & y_1 & y_2 & s_1 & s_2 & s_3 & s_4 & \text{RHS} & \alpha \\
\hline
s_1 & 1 & 1 & 1 & 0 & 0 & 0 & 1 &  \\
s_2 & 1.2 & 0.5 & 0 & 1 & 0 & 0 & 1 &  \\
s_3 & 1 & 0 & 0 & 0 & 1 & 0 & 1 &  \\
s_4 & 0 & 1 & 0 & 0 & 0 & 1 & 1 &  \\
\hline
\text{obj} & -1.2 & -1 & 0 & 0 & 0 & 0 & - & - \\
\end{array}
\]
we can define the slack variables equal to the right hand side, and this is in turn a basic feasible solution, so we can jump into phase 2.

We select the $y_1$ as the entering variable and calculate the alpha value for each basic variable
\[
\begin{array}{c|cccccc|c|c}
\text{Basic Var} & y_1 & y_2 & s_1 & s_2 & s_3 & s_4 & \text{RHS} & \alpha \\
\hline
s_1 & 1 & 1 & 1 & 0 & 0 & 0 & 1 & \frac{1}{1} \\
s_2 & 1.2 & 0.5 & 0 & 1 & 0 & 0 & 1 & \frac{1}{1.2} \\
s_3 & 1 & 0 & 0 & 0 & 1 & 0 & 1 & \frac{1}{1} \\
s_4 & 0 & 1 & 0 & 0 & 0 & 1 & 1 & \frac{1}{0} \\
\hline
\text{obj} & -1.2 & -1 & 0 & 0 & 0 & 0 & - & - \\
\end{array}
\]
We pivot this on the 1st column ($y_1$) and the 2nd row ($s_2$)
\[
\begin{array}{c|cccccc|c|c}
\text{Basic Var} & y_1 & y_2 & s_1 & s_2 & s_3 & s_4 & \text{RHS} & \alpha \\
\hline
s_1 & 0 & \frac{7}{12} & 1 & -\frac{5}{6} & 0 & 0 & \frac{1}{6} &  \\
y_1 & 1 & \frac{5}{12} & 0 & \frac{5}{6} & 0 & 0 & \frac{5}{6} &  \\
s_3 & 0 & -\frac{5}{12} & 0 &  -\frac{5}{6} & 1 & 0 & \frac{1}{6} &  \\
s_4 & 0 & 1 & 0 &  0 & 0 & 1 & 1 &  \\
\hline
\text{obj} & 0 & -0.5 & - & - & - & - & - & - \\
\end{array}
\]
With blands rule, we pick $y_2$ and calculate the alpha value for each basic variable.
\[
\begin{array}{c|cccccc|c|c}
\text{Basic Var} & y_1 & y_2 & s_1 & s_2 & s_3 & s_4 & \text{RHS} & \alpha \\
\hline
s_1 & 0 & \frac{7}{12} & 1 & -\frac{5}{6} & 0 & 0 & \frac{1}{6} & \frac{2}{7} \\
y_1 & 1 & \frac{5}{12} & 0 & \frac{5}{6} & 0 & 0 & \frac{5}{6} & \frac{2}{1} \\
s_3 & 0 & -\frac{5}{12} & 0 &  -\frac{5}{6} & 1 & 0 & \frac{1}{6} & -\frac{1}{5} \\
s_4 & 0 & 1 & 0 &  0 & 0 & 1 & 1 & \frac{1}{1} \\
\hline
\text{obj} & 0 & -0.5 & - & - & - & - & - & - \\
\end{array}
\]
We pivot on the 2nd column ($y_2$) and the 1st row ($s_1$).
\[
\begin{array}{c|cccccc|c|c}
\text{Basic Var} & y_1 & y_2 & s_1 & s_2 & s_3 & s_4 & \text{RHS} & \alpha \\
\hline
y_2 & 0 & 1 & \frac{12}{7} & -\frac{10}{7} & 0 & 0 & \frac{2}{7} &  \\
y_1 & 1 & 0 & -\frac{5}{7} & \frac{60}{42} & 0 & 0 & \frac{30}{42} &  \\
s_3 & 0 & 0 & \frac{5}{7} &  -\frac{60}{42} & 1 & 0 & \frac{12}{42} &  \\
s_4 & 0 & 0 & -\frac{12}{7} &  \frac{60}{42} & 0 & 1 & \frac{30}{42} &  \\
\hline
\text{obj} & 0 & 0 & 0.857 & 0.286  & 0 & 0 & 0 & -1.143 \\
\end{array}
\]
This is the optimal solution to the LP relaxed problem.


\subsection{Part b}
Solve the above problem with the branch and bound method by enumerating nodes in the tree and solving the LP subproblems using GAMS.

\textbf{Solution: }

\end{document}