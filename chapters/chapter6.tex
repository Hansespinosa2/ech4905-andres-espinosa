\section{Chapter 6 - Optimality Conditions}
\subsection{Tuesday 03/04/2025}
Considering an optimization problem with only equality constraints as such

\begin{align}
  \text{minimize} & \quad f(x) \\
  \text{subject to} & \quad h(x) =0
\end{align}
We have a lagragian function 
\begin{equation}
    L(x, \nu) = f(x) + \nu h(x)
\end{equation}
where
\begin{equation}
    \nabla_x L = \nabla f + \nu \nabla h = 0
\end{equation}
The geometric interpretation of the equation above is that the gradient of the objective function is directly opposite of the gradient of the equality constraint.

If we have multiple equality constraints, the problem will look like

\begin{align}
  \text{minimize} & \quad f(x) \\
  \text{subject to} & \quad h_k(x) = 0, \quad \forall k \in [1,\dots,K]
\end{align}

The lagrangian then looks like 
\begin{equation}
    L(x,\nu) = f(x) + \nu^T h(x)
\end{equation}
The solution to the lagrangian can be found by solving the following system of equations.
\begin{gather}
    \frac{d L}{d \textbf{x}} = \textbf{0} \\
    \frac{d L}{d \nu} = \nabla h(\textbf{x}) 
\end{gather}

This can be extended to equality and inequality constrained problems in the KKT conditions.
For the KKT conditions to apply, they must pass constraint qualifications.
Generally, the KKT conditions apply to any NLP written in the form

\begin{align}
  \text{minimize} & \quad f(x) \\
  \text{subject to} & \quad g_i (x) \leq 0, \quad \forall i \in [1,\dots, J] \\
  & \quad h_k(x) = 0, \quad \forall k \in [1,\dots,K] 
\end{align}
As we introduce inequalities to the problem, there are some that are inactive and some that are active.
So, we construct a similar lagragian that handles and ignores inactive constraints.

\begin{equation}
    L (x, \lambda, \nu) = f(x) + \lambda^\top g(x) + \nu^\top h(x)
\end{equation}
We take the following gradients
\begin{gather}
    \nabla_x L(x, \lambda, \nu) = 0 = \nabla f(x) + \lambda^\top \nabla g(x) + \nu^\top \nabla h(x) \\
    \nabla_\lambda L(x, \lambda, \nu) = g(x) \leq 0 \\
    \nabla_\nu L(x, \lambda, \nu) = h(x) = 0 \\ 
    \lambda_i g_i(x) = 0 \\
    \lambda \geq 0 
\end{gather}

In order for these KKT conditions to be applied, a problem must pass some constraint qualification criteria.
For example, the gradients are linearly independent is a qualification criteria for the constraints.

Consider the following optimization problem

\begin{align}
  \text{minimize} & \quad x_1^2 - x_2 \\
  \text{subject to} & \quad x_1 - 1 \geq 0 \\
  & \quad x_1 + x_2^2 \leq 26 \\
  & \quad x_1 + x_2 = 6
\end{align}
To derive the KKT conditions, we flip the first inequality to be less than and bring the constants to the other side.
\begin{align}
    \text{minimize} & \quad x_1^2 - x_2 \\
    \text{subject to} & \quad -x_1 + 1 \leq 0 \\
    & \quad x_1 + x_2^2 -26 \leq 0 \\
    & \quad x_1 + x_2 -6 = 0
\end{align}
The lagrangian for this problem is 

\begin{equation}
    L(x, \lambda, \nu) = x_1^2 - x_2 + \lambda_1 (-x_1 + 1) + \lambda_2 (x_1 + x_2^2 - 26) + \nu(x_1 + x_2 - 6)
\end{equation}

The gradients that we need for the KKT conditions are 

\begin{align}
    \nabla_x f(x) = 
  \begin{bmatrix}
     2x_1 \\ -1
  \end{bmatrix},
  \quad
  \nabla_x g_1(x) = 
  \begin{bmatrix}
    -1 \\ 0
  \end{bmatrix},
  \quad
  \nabla_x g_2(x) = 
  \begin{bmatrix}
    1 \\ 2x_2
  \end{bmatrix},
  \quad
  \nabla_x h(x) = 
  \begin{bmatrix}
    1 \\ 1
  \end{bmatrix}
\end{align}

The KKT conditions are then
\begin{align}
  \begin{bmatrix}
     2x_1 \\ -1
  \end{bmatrix} 
  +
  \lambda_1
  \begin{bmatrix}
    -1 \\ 0
  \end{bmatrix}
  +
  \lambda_2
  \begin{bmatrix}
    1 \\ 2x_2
  \end{bmatrix}
  +
  \nu
  \begin{bmatrix}
    1 \\ 1
  \end{bmatrix}
  = 0 
  \\
  x_1 + x_2 - 6 = 0
  \\
  -x_1 + 2 \leq 0, \quad
   x_1 + x_2^2 - 26 \leq 0
  \\
  \lambda_1 (-x_1 + 2) = 0, \quad 
  \lambda_2 (x_1 + x_2 -6) = 0 
  \\
  \lambda_1, \lambda_2 \geq 0
\end{align}