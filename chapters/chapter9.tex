\section{Solving MILPs}
\subsection{Branch and Bound}
The first step of the branch and bound algorithm to solve a MILP is to first solve its LP relaxation.
We then go through branching to identify integers that are fractional.
We create two subproblems, one which is concerned with the left branch and one that is concerned with the right branch.

\subsubsection{Example Problem}
Consider an example LP relaxation problem written below:

\begin{align}
  \text{minimize} & \quad  86y_1 + 4y_2 + 40y_3\\
  \text{subject to} & \quad 744 y_1 + 76y_2 + 42y_3 \leq 875 \\
  & \quad 67y_1 + 27y_2 + 53y_3 \leq 875 \\
  & \quad 0 \leq y_1, y_2, y_3 \leq 1
\end{align}

The optimal solution to this relaxed problem is 
$y_1, y_2,y_3 = 1, 0.776, 1; f = 129.6$

Since $y_2$ is a decimal, we branch it to $y_2 = 0,1$ and solve the LP relaxation of two subproblems.
When $y_2 = 0$, we get the solution $(1,0,1); f = 126$. 
When $y_2=1$, we get the solution $(0.978,1,1); f = 128.11$.
This improves our bounds because we get the lower bound of $126$ and upper bound of 128.11.
Our worst LP relaxed optimal value is 128.11 and our best feasible optimal value is 126.

\subsection{Cutting Planes}
In order to simplify and reduce the feasible region, we can employ cutting planes that introduce constraints that reduce the feasible search space.
\subsubsection{Gomory}
Given a simplex tableau with the following parameters

\begin{align}
  A = 
  \begin{bmatrix}
     B & N
  \end{bmatrix}, 
  \quad
  x =
  \begin{bmatrix}
    x_b \\ x_n
  \end{bmatrix}
\end{align}

we can derive the constraints
\begin{align}
  x_j + \sum_{i \in N} \sqcup{\alpha_{i,j}} x_j \leq \sqcup{x^*_{B,j}}
\end{align}

\subsubsection{Example}
take the IP optimization problem
\begin{align}
  \text{minimize} & \quad x_1 - 2x_2 \\
  \text{subject to} & \quad  -4x_1 + 6x_2 \leq 5 \\
  & \quad x_1 + x_2 \leq 5 \\
  & \quad x_1, x_2 \geq 0 \\
  & \quad x_1, x_2 \in \mathbb{N}
\end{align}

we can relax this problem into an LP
\begin{align}
  \text{minimize} & \quad x_1 - 2x_2 \\
  \text{subject to} & \quad  -4x_1 + 6x_2 \leq 5 \\
  & \quad x_1 + x_2 \leq 5 \\
  & \quad x_1, x_2 \geq 0 \\
  & \quad x_1, x_2 \in \mathbb{R}
\end{align}

Transforming this problem into slack form:
\begin{align}
  \text{minimize} & \quad x_1 - 2x_2 \\
  \text{subject to} & \quad  -4x_1 + 6x_2 + x_3 = 5 \\
  & \quad x_1 + x_2 + x_4 = 5 \\
  & \quad x_1, x_2, x_3, x_4 \geq 0 \\
  & \quad x_1, x_2, x_3, x_4 \in \mathbb{R}
\end{align}

The tableau that comes from this is:
\[
\begin{array}{c|cccc|c}
\text{Basic} & x_1 & x_2 & x_3 & x_4 & \text{RHS} \\
\hline
x_3 & -4 & 6 & 1 & 0 & 5 \\
x_4 & 1 & 1 & 0 & 1 & 5 \\
\hline
\text{Z} & 1 & -2 & 0 & 0 & 0 \\
\end{array}
\]

We get the solved simplex
\[
\begin{array}{c|cccc|c}
\text{Basic} & x_1 & x_2 & x_3 & x_4 & \text{RHS} \\
\hline
x_2 & 0 & 1 & 0.1 & 0.4 & 2.5 \\
x_1 & 1 & 0 & -0.1 & 0.6 & 2.5 \\
\hline
\text{Z} & 1 & -2 & 0 & 0 & 0 \\
\end{array}
\]

From this, we can get the constraints for the cutting planes by doing the following:

\begin{align}
  x_2 + \sqcup{0.1} x_3 + \sqcup{0.4} x_4 \leq \sqcup{2.5} \\
  x_1 + \sqcup{-0.1} x_3 + \sqcup{0.6} x_4 \leq \sqcup{2.5} \\
  = \\
  x_2  \leq 2 \\
  x_1 - x_3 \leq \sqcup{2}
\end{align}
We introduce the cuts on whichever $x_i$ of the LP relaxed solution variables are not integers.
So, we at most introduce $m$ constraints, the number of original constraints in the problem.

\subsubsection{Another Example}

\begin{align}
  \text{minimize} & \quad 13x_1 - 13x_2 \\
  \text{subject to} & \quad 2x_1 + 9x_2 + x_3 = 29 \\
  & \quad 11x_1 - 8x_2 + x_4 = 79 \\
  & \quad x_i \geq 0 \\
  & \quad x_i \in \mathbb{N} 
\end{align}

This simplex can be solved to be 
\[
\begin{array}{c|cccc|c}
\text{Basic} & x_1 & x_2 & x_3 & x_4 & \text{RHS} \\
\hline
x_2 & 0 & 1 & 0.1 & -0.002 & 4 \\
x_1 & 1 & 0 & 0.04 & 0.08 & 8.2 \\
\hline
\text{Z} & 0 & 0 & 1.45 & 0.01 & 4.8 \\
\end{array}
\]
We then can solve and get the Gomory cuts from this problem.

\subsection{MINLP}
The general MINLP problem below

\begin{align}
  \text{minimize} & \quad f(x,y) \\
  \text{subject to} & \quad h(x,y) = 0 \\
  & \quad g(x,y) \leq 0 \\
  & \quad x \in \mathbb{R}^n, y \in \{ 0,1 \}^m
\end{align}

This can be turned into another type of MINLP (not always, but generally, that are easier to solve)

\begin{align}
  \text{minimize} & \quad c^\top y + f(x) \\
  \text{subject to} & \quad h(x,y) = 0 \\
  & \quad g(x) + By \leq 0 \\
  & \quad Ay \leq a \\
  & \quad x \in X, y \in \{ 0,1 \}^m
\end{align}

Branch and bround is typically difficult to apply on these types of problems.
We also have decomposition approach where we fix the integer and continuous variables and solve the NLP to get the lower bound and MILP to get the upper bound.
They then ideally converge to the same value.

\subsection{Piecewise linearization}
We can take MINLPs and then apply piecewise linearization to approximate the function.